%! Author = Alexander Lesnov
%! Date = 09.04.2023

% Document
\section{Методы машинного обучения}\label{sec:machine_learning_techniques.tex}

        В данной главе мы рассмотрим различные методы машинного обучения, которые используются для решения задач классификации, регрессии, кластеризации и т.д.

    Одним из самых простых методов машинного обучения является линейная регрессия. Его можно использовать для построения модели, которая предсказывает значение зависимой переменной на основе нескольких независимых переменных. Линейная регрессия определяет коэффициенты для каждой независимой переменной, которые максимизируют качество предсказаний.

    Более сложным методом машинного обучения является метод опорных векторов. Он используется для решения задач классификации и регрессии и основан на поиске гиперплоскости, которая максимально разделяет классы или предсказывает значение зависимой переменной.

    Другим методом машинного обучения, который часто используется в задачах классификации, является метод k-ближайших соседей. Он основан на поиске k ближайших объектов к заданному объекту и определении класса этого объекта на основе классов ближайших соседей.

    Также в машинном обучении широко используются нейронные сети ~\cite{Goodfellow-et-al-2016}, которые моделируют работу человеческого мозга и могут обучаться на большом количестве данных.

        $$
        y_j^{(i+1)} = f\left(\sum_{k=1}^{K} w_{j,k}^{(i)} y_k^{(i)} + b_j^{(i)}\right)
        $$


    едставлены только некоторые методы машинного обучения, их существует гораздо больше. Выбор метода зависит от конкретной задачи, которую необходимо решить, и от доступных данных.

    Центральной идеей машинного обучения является минимизация ошибки предсказания на тестовых данных. Для этого необходимо правильно подобрать параметры модели и подготовить данные для обучения. Важной частью машинного обучения является оценка качества модели на тестовых данных, чтобы избежать переобучения.

    Одним из популярных инструментов для реализации методов машинного обучения является язык программирования Python и библиотеки для машинного обучения, такие как TensorFlow, PyTorch и Scikit-learn.

    Методы машинного обучения - в этой секции вы можете описать различные методы машинного обучения, такие как регрессия, классификация, кластеризация и т.д.

